
\documentclass[12pt]{report}
\usepackage[utf8]{inputenc}
\usepackage[czech]{babel}
\usepackage[IL2]{fontenc}
\usepackage{listings}
\usepackage{graphicx}

\begin{document}
\begin{titlepage} 
\begin{figure}[!ht] %ht?? 
\includegraphics[width=.4\textwidth]{logo.png}
\label{fig: sez} %odkaz
\end{figure}
	\newcommand{\HRule}{\rule{\linewidth}{0.5mm}} 	
	\center 
	\textsc{\Large KIV/WEB}\\[0.5cm] 
	\HRule\\[0.4cm]
	
	{\huge\bfseries Semestrální práce, Webová konference}\\[0.4cm]
	
	\HRule\\[1.5cm]
	{\large\textit{Autor}}\\
	 \textsc{Jan Rychlík, A16B0125,rychlikj@students.zcu.cz} 
	\vfill\vfill\vfill 
	{\large\today} 
	\vfill 
	
\end{titlepage}
\tableofcontents
\chapter{Zadání}
\section{Popis zadání}
Standartním zadáním semestrální práce z WEBu bylo vytvořit webové stránky na námi zvolené téma. 
\section{Popis činosti webových stránek}
\begin{itemize}
\item Uživateli systému budou autoři příspěvků (vkládají abstrakty a PDF dokumenty), recenzenti příspěvků (hodnotí příspěvky) a administrátoři (spravují uživatele, přiřazují příspěvky recenzentům a rozhodují o publikování příspěvků). Každý uživatel se bude do systému přihlašovat prostřednictvím uživatelského jména a hesla. 
\item Nepřihlášený uživatel vidí pouze publikované příspěvky.
\item Nový uživatel se bude moci zaregistrovat, čímž získá status autora.
\item Přihlášený autor vidí svoje příspěvky a stav, ve kterém se nacházejí (v recenzním řízení / přijat +hodnocení / odmítnut +hodnocení). Příspěvky může přidávat, editovat a volitelně i mazat.
\item Přihlášený recenzent vidí příspěvky, které mu byly přiděleny k recenzi, a může je hodnotit (alespoň 3 kritéria). Pokud příspěvek nebyl dosud schválen, tak své hodnocení může změnit.
\item Administrátor spravuje uživatele (určuje jejich role a může uživatele zablokovat či smazat), přiřazuje neschválené příspěvky recenzentům k ohodnocení (každý příspěvek bude recenzován minimálně třemi recenzenty) a na základě recenzí rozhoduje o přijetí nebo odmítnutí příspěvku. Přijaté příspěvky jsou automaticky publikovány ve veřejné části webu.
\item Databáze musí obsahovat alespoň 3 tabulky dostatečně naplněné daty pro předvedení funkčnosti aplikace.
\end{itemize}
\section{Nutné požadavky semestrální práce}
\begin{itemize}
\item  Práce musí být osobně předvedena cvičícímu a po schválení odevzdána na CourseWare či Portál.
\item  K práci musí být dodána dokumentace (viz dále) a skripty pro instalaci databáze (např. získané exportem databáze).
\item   Web musí dodržovat MVC architekturu.
\item    Pro práci s databází musí být využito PDO nebo jeho ekvivalent a používány předpřipravené dotazy (prepared statements).
\item    Web musí obsahovat responzivní design.
\item    Web musí obsahovat ošetření proti základním typům útoku (XSS, SQL injection).
\item    Web musí fungovat i s "ošklivými" URL adresami.
\item    Aplikaci není možné realizovat s využitím PHP frameworků (zakázáno např. Nette, Symfony atd.).
\item    Front-end je vhodné realizovat s využitím frameworku Bootstrap (getbootstrap.com).
\end{itemize}

\chapter{Popis použitých tehnologií}
Základem semestrální práce je struktura MVC. Použité technologe webu jsou: HTML, CSS, PHP,Twig, JavaScript a MySQL. Bootstrap byl použit pro responzivní vzhled strákny a rozmístění prvků na stránce.

\section{HTML}
\textbf{HTML5} - vzhled stránky, složka view.

\section{CSS}
\textbf{CSS3} - prace s HTML, pouze ve složce css.

\section{PHP}
\textbf{PHP} - ve složce models užito pro komunikaci s databazí, ve složce controllers na testování vstupů a dokumentace v kódu.

\section{MySQL}
\textbf{MySQL}- vytvoření databáze.

\section{Twig}
\textbf{Twig} - složka sablon, sablon.html (obsahuje odkazy na používané technologie).

\section{JavaScript}
\textbf{JavaScript}- zobrazení minimapy, v  záložce Kontakty.

\section{Bootstrap}
\textbf{Bootstrap} - spravuje vzhled stránky, nejvíce souvisí se složkou view




\chapter{Adresářová struktura}


\section{controllers}
\textbf{controllers} - obsahuje controllery, které získávají data z modelů.


\section{css}
\textbf{css} -Css styly aplikace.

\section{Models}
\textbf{models} - všechny kódy k práci s databazí, včetně logických operací.

\section{Rescouse}
\textbf{Rescouse} - složka, kam se ukladají pdf soubory nahrané na stránce.

\section{Sablon}
\textbf{Sablon} - ve složce je kód, který určuje vzhled stránky, včetně odkatů na použité prvky z webu.

\section{views}
\textbf{views} - složka, kde jsou uloženy jednotlivé kódy k zobrazení jednotlivých částí semestrální práce.




\chapter{Architektura aplikace}
\section{Controllers}
Obsahuje tři třídy \textbf{homeController, newsController a useControler}
\newline
\textbf{homeController}  -zobrazuje titulní stránku s příspevky. \newline
\textbf{newsController} -  zobrazuje příspěvky podle přihlášeného uživatele.\newline
\textbf{userController}  - kontroluje přihlášení uživatele. K administrátorovy se chová stejně jako k uživateli s tím že, v potřebných funkcích má adminstrátor přidanou vyjímku.

\section{Models} 
Obsahuje dvě třídy \textbf{db a phpWrapper}. 
Provádí se zde veškeré logické operace, včetně načítání z databází. 
\section{view} 
Obsahuje několik tříd, určených pro vzhled stránky.

\chapter{Závěr}
Semestrální práce splňuje všechny povinné části zadání a některé nepovinné(např. práje je uložena na \textbf{Github}). Uživatel se může registrovat nebo přihlásit, poté může přidavat, hodnotit nebo mazat některé příspěvky. Administrátor může evidovat uživatele, blokovat je nebo mazat. U příspěvků může rozhodnout, zda ho smazat, určit kdo ho bude hodnotit nebo určit, zda se zobrazí na fóru. 






\end{document}
























































